% Define block styles
\tikzset{
   papDecision/.style = {
         diamond,
         draw, 
         text width = 20 mm, 
         align = center, 
         text badly centered,
         inner sep = 1 pt,
         font=\ttfamily\footnotesize,
         %line width = 1,
         minimum width = 30mm,
         minimum height = 7mm,
      },
   papStart/.style = {
         rectangle,
         draw, 
         align = center, 
         text width = 3cm, 
         text badly centered,
         inner sep = 4 pt,
         rounded corners=10pt,
         font=\ttfamily\footnotesize,
         %line width = 1,
         minimum width = 30mm,
         minimum height = 7mm,
      },
   papEnd/.style = {
         rectangle,
         draw, 
         align = center, 
         text width = 3cm, 
         text badly centered,
         inner sep = 4 pt,
         rounded corners=10pt,
         font=\ttfamily\footnotesize,
         %line width = 1,
         minimum width = 30mm,
         minimum height = 7mm,
      },
   papData/.style = {
         trapezium,
         draw, 
         align = center, 
         text width = 20 mm, 
         text badly centered,
         inner sep = 4 pt,
         trapezium left angle=70,
         trapezium right angle=110,
         font=\ttfamily\footnotesize,
         %line width = 1,
         minimum width = 30mm,
         minimum height = 7mm,
      },
   papPredProc/.style = {
         draw,
         rectangle split,
         rectangle split horizontal,
         rectangle split parts = 3,
         rectangle split empty part width=-8pt,
         align = center, 
 %       text width = 4.5 em, 
         text badly centered,
 %        inner sep = 4 pt,
         font=\ttfamily\footnotesize,
         %line width = 1,
         minimum width = 30mm,
         minimum height = 7mm,
      },
   papProcess/.style = {
         rectangle,
         draw,
         align = center, 
         text width = 3cm, 
         text badly centered,
         %inner sep = 2 pt,
         font=\ttfamily\footnotesize,
         %line width = 1,
         minimum width = 30mm,
         minimum height = 7mm,
      },
   papLine/.style = {
         draw,
         -stealth,
         font=\ttfamily\footnotesize,
         %line width = 1,
      },
}

\newcommand{\papYes}{ja}
\newcommand{\papNo}{nein}


%%%%%%%%%%%%%%%%%%%%%%%%%%%%%%%%%%%%%%%%%%%%%%%%%%%%%%%%%%%%%%%%%


\begin{figure}
\centering
\begin{tikzpicture}[node distance = 2cm, auto]

% Place nodes
\node [papStart] (Start1){Start};
\node [papProcess, below of = Start1] (pro1){Warte auf button};
\node [papDecision, below of = pro1, yshift= -9mm](dec1){Button gedrückt?};
\node [papProcess, below of = dec1] (pro2){Starte Timer};
\node [papProcess, below of = pro2] (pro3){Warte auf button};
\node [papDecision, below of = pro3, yshift= -9mm](dec2){Button losgelassen?};
\node [papProcess, below of = dec2] (pro4){Stope Timer};
\node [papProcess, below of = pro4] (pro5){Summiere Timer auf};
\node [papDecision, below of = pro5, yshift= -9mm](dec3){Button 3x gedrückt?};
\node [papProcess, below of = dec3] (pro6){Dividiere Summe durch 3};
\node [papEnd, below of = pro6] (End) {Starte Metronom};


% Draw edges
\path [papLine] (Start1) -- (pro1);
\path [papLine] (pro1) -- (dec1);
\path [papLine] (dec1.east) -| node [right] {\papNo}(2, -2) -- (pro1.east);
\path [papLine] (dec1) -- node [right] {\papYes} (pro2);
\path [papLine] (pro2) -- (pro3);
\path [papLine] (pro3) -- (dec2);
\path [papLine] (dec2.east) -| node [right] {\papNo}(2, -8.9) -- (pro3.east);
\path [papLine] (dec2) -- node [right] {\papYes} (pro4);
\path [papLine] (pro4) -- (pro5);
\path [papLine] (pro5) -- (dec3);
\path [papLine] (dec3.east) -| node [right] {\papNo}(4, -2) -- (pro1.east);
\path [papLine] (dec3) -- node [right] {\papYes} (pro6);
\path [papLine] (pro6) -- (End);

\end{tikzpicture}
\caption{Programmablauf} \label{fig:pgmAblauf}
\end{figure}